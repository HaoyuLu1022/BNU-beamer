%----------------------------------------------------------------------------------------
%	PACKAGES AND THEMES
%----------------------------------------------------------------------------------------
\documentclass[aspectratio=169,xcolor=dvipsnames, t]{beamer}
\usepackage{fontspec} % Allows using custom font. MUST be before loading the theme!
\usetheme{SimplePlusAIC}
\usepackage{ulem} 
\usepackage{hhline}
\usepackage{hyperref}
\usepackage{graphicx} % Allows including images
\usepackage{booktabs} % Allows the use of \toprule, \midrule and  \bottomrule in tables
\usepackage{svg} %allows using svg figs
\usepackage{tikz}
\usepackage{makecell}
\usepackage{amsmath} % For \textsubscript command
% \usepackage[english]{babel}
\usepackage[backend=biber, 
style=gb7714-2015,
doi=false,%取消显示doi
url=false,%取消显示原文链接
gbpunctin=false, 
]{biblatex}
\usepackage{wrapfig}
% ADD YOUR PACKAGES BELOW
\usetikzlibrary{calc,patterns.meta}
\usepackage{totcount}
\regtotcounter{section}
\usepackage{forest}
\usepackage{colortbl}
\usepackage{color,soul}
\usepackage{xcolor}
\usepackage{nameref}
\usepackage[UTF8]{ctex}
\addbibresource{test-ref.bib}
\AtBeginBibliography{\tiny}
\usepackage{caption}
\captionsetup{labelformat=simple}
\usepackage{multicol}
\usepackage{multirow}
\usepackage{makecell}
\usepackage{subcaption}
\captionsetup[subfigure]{labelformat=empty, justification=centering}
\usepackage[ruled, vlined, linesnumbered]{algorithm2e}
\usepackage{overpic}
% Redefinition of the \section command so that each one is labeled \label{sec:n} where n is its index 
\let\oldsection\section
\renewcommand{\section}[2][\relax]{%
    \ifx#1\relax
      \oldsection{#2}%
    \else
      \oldsection[#1]{#2}%
    \fi%
    \label{sec:\thesection}%
}

% Definition of custom colors based on the initial figure of the bar by the OP
\definecolor{myblue}{HTML}{57AED1}
\definecolor{mygreen}{HTML}{8BC53F}
\definecolor{mygray}{HTML}{DDDDDD}
\definecolor{BlueIuss}{HTML}{033354}
\definecolor{LightBlueIuss}{HTML}{1b4a74}

% Definition of custom tikz styles in order to ease readability
\tikzset{
    % Bar style (Argument : color)
    sectionbar/.style={
        % Filling with one color as a preaction, in order to avoid reset by the pattern color
        preaction={fill=#1!70},
        % Application of the line pattern on to of the fill
        pattern={Lines[angle=45,distance={6pt},line width=3pt]},pattern color=#1
    },
    % Node style (Arguments : color, section number)
    sectionnode/.style 2 args={
        fill=#1,
        draw=white,
        thick,
        circle,
        text=white,
        radius=10pt,
        % Display of the section name below the cicle
        label={[text=#1, text width=2cm, align=center]below:\nameref{sec:#2}},
        }
}


% Actual definition of the colorbar based on Gonzalo Medina's initial proposal
\makeatletter
    \def\pbar@progressbar{} % the progress bar
    \newcount\pbar@tmpcnta% auxiliary counter
    \newcount\pbar@tmpcntb% auxiliary counter
    \newdimen\pbar@pbht %progressbar height
    \newdimen\pbar@pbwd %progressbar width
    \newdimen\pbar@tmpdim % auxiliary dimension
    \pbar@pbwd=\linewidth
    \pbar@pbht=4pt

% The progress bar
\def\pbar@progressbar{%
    \pbar@tmpcnta=\value{section} % tmpcnta stores the section number
    \pbar@tmpcntb=\totvalue{section} % tmbcountb sotres the total amount of sections
    \advance\pbar@tmpcntb by 1 % tmbcountb is advanced by 1 in order to have the last bar segment after the last node

    \begin{tikzpicture}[very thin]
        % Clipping scope to avoid tests for the bar dimensions
        \begin{scope}
        % Clipping path
        \path[rounded corners=2pt,clip] (0pt,{-\pbar@pbht/2}) rectangle (\pbar@pbwd,{\pbar@pbht/2});
        % Gray bar (from 0 to last section)
        \path[sectionbar=mygray] (0pt,{-\pbar@pbht/2}) rectangle (\linewidth,{\pbar@pbht/2});
        % Blue bar (from 0 to the current section)
        \path[sectionbar=BlueIuss] (0pt,{-\pbar@pbht/2}) rectangle ({(\pbar@tmpcnta-0.5)*\linewidth/\pbar@tmpcntb},{\pbar@pbht/2});
        % Green bar (from current to next section)
        \path[sectionbar=LightBlueIuss] ({(\pbar@tmpcnta-0.5)*\linewidth/\pbar@tmpcntb},{-\pbar@pbht/2}) rectangle ({(\pbar@tmpcnta+0.5)*\linewidth/\pbar@tmpcntb},{\pbar@pbht/2});
        \end{scope}
        % Drawing of the nodes on top of the bars, based on the number of the current section
        \foreach \secnumber in {1,...,\totvalue{section}}{
            % Number is lower, section is past, blue color
            \ifnum\secnumber<\pbar@tmpcnta
                \node[sectionnode={BlueIuss}{\secnumber}] at ({(\secnumber-0.5)*\linewidth/\pbar@tmpcntb},0) {\strut\secnumber};
            \fi
            % Number is equal, section is current, green color
            \ifnum\secnumber=\pbar@tmpcnta
                \node[sectionnode={LightBlueIuss}{\secnumber}] at ({(\secnumber-0.5)*\linewidth/\pbar@tmpcntb},0) {\strut\secnumber};
            \fi
            % Number is larger, to be done section, gray color
            \ifnum\secnumber>\pbar@tmpcnta
            \node[sectionnode={mygray}{\secnumber}] at ({(\secnumber-0.5)*\linewidth/\pbar@tmpcntb},0) {\strut\secnumber};
            \fi
        }
  \end{tikzpicture}%
}

\addtobeamertemplate{headline}{}
{%
  \begin{beamercolorbox}[wd=\paperwidth,ht=10ex,center,dp=-10ex]{white}%
    \pbar@progressbar%
  \end{beamercolorbox}%
}
\makeatother

\setbeamertemplate{frametitle continuation}{}
%----------------------------------------------------------------------------------------
%	TITLE PAGE CONFIGURATION
%----------------------------------------------------------------------------------------

\title[BNU-beamer]{BNU Beamer模板} % The short title appears at the bottom of every slide, the full title is only on the title page
% \subtitle{Subtitle}


\author{Halve Luve}
\institute[BNU-AI]{北京师范大学人工智能学院}
% Your institution as it will appear on the bottom of every slide, maybe shorthand to save space

\date{\today} % Date, can be changed to a custom date

%----------------------------------------------------------------------------------------
%	PRESENTATION SLIDES
%----------------------------------------------------------------------------------------

\begin{document}
\maketitlepage
%\section{Overview}
\begin{frame}[t]{目录}
    % Throughout your presentation, if you choose to use \section{} and \subsection{} commands, these will automatically be printed on this slide as an overview of your presentation
    \tableofcontents
\end{frame}

\makesection{部分一}
    \begin{frame}{无序列表}
        以下是无序列表的例子:
        \begin{itemize}
            \item 第一点
            \item 第二点
        \end{itemize}
    \end{frame}

    \begin{frame}{分列展示}
        \begin{columns}
            \column{.5\textwidth}
            这是第一列
            
            \column{.5\textwidth}
            这是第二列
        \end{columns}
    \end{frame}

\makesection{部分二}
    \begin{frame}{表格}
        \begin{table}
            \centering
            \begin{tabular}{ccc}
                \hline
                \textbf{标签1} & \textbf{标签2} & \textbf{标签3} \\
                \hline 
                值1 & 值2 & 值3 \\ 
                值4 & 值5 & 值6 \\ 
                值7 & 值8 & 值9 \\ 
                \hline
            \end{tabular}
            \caption{表1的题注}
            \label{tab:table1}
        \end{table}
    \end{frame}

    \begin{frame}{图像}
        \begin{figure}[H]
            \centering
            \includegraphics[width=.5\textwidth]{example-image-a}
            \caption{图1的题注}
            \label{fig:figure1}
        \end{figure}
    \end{frame}

    \begin{frame}{引用}
        这里引用了一张图 (图\ref{fig:figure1})\\
        这里引用了一个表 (表\ref{tab:table1})\\
        这里引用了一篇文章 \cite{vaswani2017attention}\\
        % 这里以脚注形式引用了一篇文章 \footfullcite{vaswani2017attention}
    \end{frame}

    \lecture{Test lecture one}{lone}
    \begin{frame}{定理}
        \begin{theorem}
            这是定理1。
        \end{theorem}
        \begin{definition}
            这是一个定义。
        \end{definition}
        \begin{corollary}
            这是一个推论。
        \end{corollary}
    \end{frame}

    \lecture{Test lecture two}{ltwo}
    \begin{frame}{公式}
        \begin{theorem}
            这是定理2。
        \end{theorem}
        这是公式块:
        \begin{equation}
            H(x)=-\sum\limits_i^Np(x_i)\log p(x_i).
        \end{equation}
        这是内联公式:$H(x)=-\sum\limits_i^Np(x_i)\log p(x_i)$。
    \end{frame}

\makesection{参考文献}

\begin{frame}[allowframebreaks]{参考文献}
    % Beamer does not support BibTeX so references must be inserted manually as below
    \newcommand{\ftcmu}{\fontspec{CMU Serif}\selectfont}
    \renewcommand{\bibfont}{\ftcmu}
    \setbeamertemplate{bibliography item}[text]
    \vspace{-20pt}
    % For compactness, 2-column layout is employed; feel free to revise or remove the multicols layout
    \begin{multicols}{2}
        \printbibliography[heading=none]
    \end{multicols}
\end{frame}

%----------------------------------------------------------------------------------------
% Final PAGE
% Set the text that is showed on the final slide
\finalpagetext{感谢倾听\quad 恳请指正}
%----------------------------------------------------------------------------------------
\makefinalpage
%----------------------------------------------------------------------------------------

\end{document}